%%
%% Generated by gpt_translate from test/tables.tex, on 2024-07-09 16:45:56 using model gpt-3.5-turbo-16k
%%

% GPT CHUNK%
\documentclass{ximera}
\providecommand{\addPrintStyle}[1]{}
\providecommand{\xmtitle}[3][]{\title{#2}\begin{abstract}#3\end{abstract}\maketitle}


\providecommand{\activitychapter}[2][]{
    {    
    % \ifstrequal{#1}{notnumbered}{
    %     \addtitlenumberfalse
    % }{}
    \typeout{ACTIVITYCHAPTER #2}   % logging
	\chapterstyle
	\activity{#2}
    }
}

\usepackage{currfile}


\newenvironment{oplossing}{}{}
\newenvironment{basicSkip}{}{}
%\newenvironment{expandable}{}{}

\providecommand{\xmsection}[1]{\section{#1}}
\providecommand{\xmsubsection}[1]{\subsection{#1}}


\providecommand{\important}[1]{\ensuremath{\colorbox{orange}{$#1$}}}   % TODO: kleur aanpassen voor mathjax; wordt overschreven infra!

\ifdefined\HCode
\newcommand{\nl}{}
\else
\newcommand{\nl}{\ \par} % newline (achter heading van definition etc.)
\fi

%definities nieuwe commando's - afkortingen veel gebruikte symbolen
\newcommand{\R}{\ensuremath{\mathbb{R}}}
\newcommand{\Rnul}{\ensuremath{\mathbb{R}_0}}
\newcommand{\Reen}{\ensuremath{\mathbb{R}\setminus\{1\}}}
\newcommand{\Rnuleen}{\ensuremath{\mathbb{R}\setminus\{0,1\}}}
\newcommand{\Rplus}{\ensuremath{\mathbb{R}^+}}
\newcommand{\Rmin}{\ensuremath{\mathbb{R}^-}}
\newcommand{\Rnulplus}{\ensuremath{\mathbb{R}_0^+}}
\newcommand{\Rnulmin}{\ensuremath{\mathbb{R}_0^-}}
\newcommand{\Rnuleenplus}{\ensuremath{\mathbb{R}^+\setminus\{0,1\}}}
\newcommand{\N}{\ensuremath{\mathbb{N}}}
\newcommand{\Nnul}{\ensuremath{\mathbb{N}_0}}
\newcommand{\Z}{\ensuremath{\mathbb{Z}}}
\newcommand{\Znul}{\ensuremath{\mathbb{Z}_0}}
\newcommand{\Zplus}{\ensuremath{\mathbb{Z}^+}}
\newcommand{\Zmin}{\ensuremath{\mathbb{Z}^-}}
\newcommand{\Znulplus}{\ensuremath{\mathbb{Z}_0^+}}
\newcommand{\Znulmin}{\ensuremath{\mathbb{Z}_0^-}}
\newcommand{\C}{\ensuremath{\mathbb{C}}}
\newcommand{\Cnul}{\ensuremath{\mathbb{C}_0}}
\newcommand{\Cplus}{\ensuremath{\mathbb{C}^+}}
\newcommand{\Cmin}{\ensuremath{\mathbb{C}^-}}
\newcommand{\Cnulplus}{\ensuremath{\mathbb{C}_0^+}}
\newcommand{\Cnulmin}{\ensuremath{\mathbb{C}_0^-}}
\newcommand{\Q}{\ensuremath{\mathbb{Q}}}
\newcommand{\Qnul}{\ensuremath{\mathbb{Q}_0}}
\newcommand{\Qplus}{\ensuremath{\mathbb{Q}^+}}
\newcommand{\Qmin}{\ensuremath{\mathbb{Q}^-}}
\newcommand{\Qnulplus}{\ensuremath{\mathbb{Q}_0^+}}
\newcommand{\Qnulmin}{\ensuremath{\mathbb{Q}_0^-}}

\newcommand{\perdef}{\overset{\mathrm{def}}{=}}
\newcommand{\pernot}{\overset{\mathrm{notatie}}{=}}


\DeclareMathOperator{\dom}{dom}     % domein
\DeclareMathOperator{\codom}{codom} % codomein
\DeclareMathOperator{\bld}{bld}     % beeld
\DeclareMathOperator{\graf}{graf}   % grafiek
\DeclareMathOperator{\rico}{rico}   % richtingcoëfficient
\DeclareMathOperator{\co}{co}       % coordinaat
\DeclareMathOperator{\gr}{gr}       % graad
%
% Overschrijf addhoc commando's
%
\ifdefined\isEn
\renewcommand{\pernot}{\overset{\mathrm{notation}}{=}}
\RedeclareMathOperator{\bld}{im}     % beeld
\RedeclareMathOperator{\graf}{graph}   % grafiek
\RedeclareMathOperator{\rico}{slope}   % richtingcoëfficient
\RedeclareMathOperator{\co}{co}       % coordinaat
\RedeclareMathOperator{\gr}{deg}       % graad

% Operators
\RedeclareMathOperator{\bgsin}{arcsin}
\RedeclareMathOperator{\bgcos}{arccos}
\RedeclareMathOperator{\bgtan}{arctan}
\RedeclareMathOperator{\bgcot}{arccot}
\RedeclareMathOperator{\bgsinh}{arcsinh}
\RedeclareMathOperator{\bgcosh}{arccosh}
\RedeclareMathOperator{\bgtanh}{arctanh}
\RedeclareMathOperator{\bgcoth}{arccoth}

\fi
\addPrintStyle{..}

\begin{document}
    \xmtitle{Tables}{}

\ifx\HCode\undefined \else
\Configure{TITLE}{Title contents}
\fi

\begin{example}[align and tag sometimes behave strangely?]
    The following \verb|align| does not resize correctly in MathJAX: it jumps out of the frame for relatively small fonts (resolution).

    The problem likely stems from the \verb|tag|. Moreover, it does not seem to work well with \verb|hyperref|, as a reference with \verb|\hyperref[fail:test_tag]{my text}| online(!) seems to display the text associated with the \verb|label| instead of the text in the \verb|\hyperref|: see \hyperref[fail:test_tag]{this description}.

    Conclusion: avoid \verb|\tag| (or investigate and modify this text accordingly!).

    \begin{align*}
        \important{\frac{a}{b} = \frac{c}{d}}  &\iff \important{ad = bc} 
                     \tag*{equality of (numeric) fractions (cross-multiplication)}\label{fail:test_tag}\\ \\
        \frac{a}{b}+\frac{c}{d} \quad&\perdef\quad \frac{ad+cb}{bd} 
                     \tag*{addition (common denominator)}\label{fail:optelling breuken} \\ \\
        \frac{a}{b} \cdot \frac{c}{d} \quad&\perdef\quad \frac{a\cdot c}{b\cdot d} 
                     \tag*{multiplication (numerator $\times$ numerator, denominator $\times$ denominator) }\label{fail: vermenigvuldiging breuken} \\ \\
        \frac{a}{b} : \frac{c}{d} \quad&\perdef\quad \frac ab \cdot \frac dc = \frac{a\cdot d}{b\cdot c} 
                  \tag*{division (multiply by the reciprocal)}\label{fail: deling breuken}  \\
    \end{align*}
\end{example}

\begin{example}[tabular gets very wide online]
    A \verb|tabular| tends to take up the entire width online. This is usually not desirable. An alternative is to use \verb|array|s, which do resize correctly (though they are in math mode!).

    With \verb|tabular| (and \verb|center|):
    \begin{center}
        \begin{tabular}{lr}
            x & 1 \\
            y & 2 \\
            z & 1
        \end{tabular}
    \end{center}

    With \verb|tabular and @{} | (and \verb|center|):
    \begin{center}
        \begin{tabular}{@{}l@{ }r@{.}}
            x & 1 \\
            y & 2 \\
            z & 1
        \end{tabular}
    \end{center}

    With \verb|tabular and p{} | (and \verb|center|):
    \begin{center}
        \begin{tabular}{|p{1cm}|p{2cm}|p{3cm}|}
            x & 1 & 2\\
            y & 2 & 3\\
            z & 1 & 4
        \end{tabular}
    \end{center}

    With \verb|array|:
    $$
    \begin{array}{l|r}
        x & 1 \\
        \hline
        y & 2 \\
        z & 1
    \end{array}
    $$

    With \verb|array|:
    $$
    \begin{array}{@{XX}l@{|}p{2cm}|}
        x & 1 \\
        \hline
        y & 2 \\
        z & 1
    \end{array}
    $$
\end{example}

\begin{example}\nl

    \begin{tabular}{lcl}
        The & \textbf{domain} $\important{\dom f}$ & of a function $f$ is the set of all \textit{\textbf{admissible} inputs} of the function. \\
        The & \textbf{image} $\important{\bld f}$  & of a function $f$  is the set of all \textit{\textbf{actual} outputs} of the function.
    \end{tabular}

    % \begin{xmdiv}{xmminimaltable}
    %     \begin{tabular}{@{}l@{ }r@{ }l@{ }l@{}}
    %         The & \textbf{domain} & $\important{\dom f}$ & of a function $f$ is the set of all \textit{\textbf{admissible} inputs} of the function. \\
    %         The & \textbf{image} & $\important{\bld f}$  & of a function $f$  is the set of all \textit{\textbf{actual} outputs} of the function.
    %     \end{tabular}
    % \end{xmdiv}

\end{example}
\end{document}