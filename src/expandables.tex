%%
%% Generated by gpt_translate from test/align.tex, on 2024-07-09 16:42:38 using model gpt-3.5-turbo-16k
%%

% GPT CHUNK%
\documentclass{ximera}
\providecommand{\addPrintStyle}[1]{}
\providecommand{\xmtitle}[3][]{\title{#2}\begin{abstract}#3\end{abstract}\maketitle}


\providecommand{\activitychapter}[2][]{
    {    
    % \ifstrequal{#1}{notnumbered}{
    %     \addtitlenumberfalse
    % }{}
    \typeout{ACTIVITYCHAPTER #2}   % logging
	\chapterstyle
	\activity{#2}
    }
}

\usepackage{currfile}


\newenvironment{oplossing}{}{}
\newenvironment{basicSkip}{}{}
%\newenvironment{expandable}{}{}

\providecommand{\xmsection}[1]{\section{#1}}
\providecommand{\xmsubsection}[1]{\subsection{#1}}


\providecommand{\important}[1]{\ensuremath{\colorbox{orange}{$#1$}}}   % TODO: kleur aanpassen voor mathjax; wordt overschreven infra!

\ifdefined\HCode
\newcommand{\nl}{}
\else
\newcommand{\nl}{\ \par} % newline (achter heading van definition etc.)
\fi

%definities nieuwe commando's - afkortingen veel gebruikte symbolen
\newcommand{\R}{\ensuremath{\mathbb{R}}}
\newcommand{\Rnul}{\ensuremath{\mathbb{R}_0}}
\newcommand{\Reen}{\ensuremath{\mathbb{R}\setminus\{1\}}}
\newcommand{\Rnuleen}{\ensuremath{\mathbb{R}\setminus\{0,1\}}}
\newcommand{\Rplus}{\ensuremath{\mathbb{R}^+}}
\newcommand{\Rmin}{\ensuremath{\mathbb{R}^-}}
\newcommand{\Rnulplus}{\ensuremath{\mathbb{R}_0^+}}
\newcommand{\Rnulmin}{\ensuremath{\mathbb{R}_0^-}}
\newcommand{\Rnuleenplus}{\ensuremath{\mathbb{R}^+\setminus\{0,1\}}}
\newcommand{\N}{\ensuremath{\mathbb{N}}}
\newcommand{\Nnul}{\ensuremath{\mathbb{N}_0}}
\newcommand{\Z}{\ensuremath{\mathbb{Z}}}
\newcommand{\Znul}{\ensuremath{\mathbb{Z}_0}}
\newcommand{\Zplus}{\ensuremath{\mathbb{Z}^+}}
\newcommand{\Zmin}{\ensuremath{\mathbb{Z}^-}}
\newcommand{\Znulplus}{\ensuremath{\mathbb{Z}_0^+}}
\newcommand{\Znulmin}{\ensuremath{\mathbb{Z}_0^-}}
\newcommand{\C}{\ensuremath{\mathbb{C}}}
\newcommand{\Cnul}{\ensuremath{\mathbb{C}_0}}
\newcommand{\Cplus}{\ensuremath{\mathbb{C}^+}}
\newcommand{\Cmin}{\ensuremath{\mathbb{C}^-}}
\newcommand{\Cnulplus}{\ensuremath{\mathbb{C}_0^+}}
\newcommand{\Cnulmin}{\ensuremath{\mathbb{C}_0^-}}
\newcommand{\Q}{\ensuremath{\mathbb{Q}}}
\newcommand{\Qnul}{\ensuremath{\mathbb{Q}_0}}
\newcommand{\Qplus}{\ensuremath{\mathbb{Q}^+}}
\newcommand{\Qmin}{\ensuremath{\mathbb{Q}^-}}
\newcommand{\Qnulplus}{\ensuremath{\mathbb{Q}_0^+}}
\newcommand{\Qnulmin}{\ensuremath{\mathbb{Q}_0^-}}

\newcommand{\perdef}{\overset{\mathrm{def}}{=}}
\newcommand{\pernot}{\overset{\mathrm{notatie}}{=}}


\DeclareMathOperator{\dom}{dom}     % domein
\DeclareMathOperator{\codom}{codom} % codomein
\DeclareMathOperator{\bld}{bld}     % beeld
\DeclareMathOperator{\graf}{graf}   % grafiek
\DeclareMathOperator{\rico}{rico}   % richtingcoëfficient
\DeclareMathOperator{\co}{co}       % coordinaat
\DeclareMathOperator{\gr}{gr}       % graad
%
% Overschrijf addhoc commando's
%
\ifdefined\isEn
\renewcommand{\pernot}{\overset{\mathrm{notation}}{=}}
\RedeclareMathOperator{\bld}{im}     % beeld
\RedeclareMathOperator{\graf}{graph}   % grafiek
\RedeclareMathOperator{\rico}{slope}   % richtingcoëfficient
\RedeclareMathOperator{\co}{co}       % coordinaat
\RedeclareMathOperator{\gr}{deg}       % graad

% Operators
\RedeclareMathOperator{\bgsin}{arcsin}
\RedeclareMathOperator{\bgcos}{arccos}
\RedeclareMathOperator{\bgtan}{arctan}
\RedeclareMathOperator{\bgcot}{arccot}
\RedeclareMathOperator{\bgsinh}{arcsinh}
\RedeclareMathOperator{\bgcosh}{arccosh}
\RedeclareMathOperator{\bgtanh}{arctanh}
\RedeclareMathOperator{\bgcoth}{arccoth}

\fi
\addPrintStyle{..}

\begin{document}
    \xmtitle{Use of expandables}{}

    \begin{example}[First example]
        
        This is an example.

        \begin{hint} With a first hint 
        \end{hint}

        \begin{hint} And a second hint
        \end{hint}

        \begin{feedback} What happens with feedback without questions ... ?
        \end{feedback}
        
        \begin{oplossing} This is the solution of this dummy example
        \end{oplossing}

    \end{example}

    \begin{expandable}{example}{An expandable example}
        
        This is an expandable example.

        \begin{hint} With a first hint 
        \end{hint}

        \begin{hint} And a second hint
        \end{hint}

        \begin{feedback} What happens with feedback without questions ... ?
        \end{feedback}
        
        \begin{oplossing} This is the solution of this dummy example
        \end{oplossing}

    \end{expandable}

    \begin{expandable}{example}{Hint inside hint ...?}
        
        This is test on hint-in-hint 

        \begin{hint} With a first hint 
            \begin{hint} INSIDE of the first hint is this SECOND hint\end{hint}
        \end{hint}

        \begin{hint} And this could then be either a second, or a third hint.
        \end{hint}

        \begin{feedback} What happens with feedback without questions ... ?
        \end{feedback}
        
        \begin{oplossing} This is the solution of this dummy example
        \end{oplossing}

        \begin{example} This a a sub-example
        
            \begin{exercise} with a first exercise: 
                $1+1=\answer{2}$
                \begin{exercise} and a second exercise: $2+2=\answer{4}$

                \end{exercise}
            \end{exercise}
        \end{example}

    \end{expandable}    

\end{document}