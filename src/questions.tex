%%
%% Generated by gpt_translate from test/questions.tex, on 2024-07-09 15:37:18 using model gpt-3.5-turbo-16k
%%

% GPT CHUNK%
\documentclass{ximera}
\providecommand{\addPrintStyle}[1]{}
\providecommand{\xmtitle}[3][]{\title{#2}\begin{abstract}#3\end{abstract}\maketitle}


\providecommand{\activitychapter}[2][]{
    {    
    % \ifstrequal{#1}{notnumbered}{
    %     \addtitlenumberfalse
    % }{}
    \typeout{ACTIVITYCHAPTER #2}   % logging
	\chapterstyle
	\activity{#2}
    }
}

\usepackage{currfile}


\newenvironment{oplossing}{}{}
\newenvironment{basicSkip}{}{}
%\newenvironment{expandable}{}{}

\providecommand{\xmsection}[1]{\section{#1}}
\providecommand{\xmsubsection}[1]{\subsection{#1}}


\providecommand{\important}[1]{\ensuremath{\colorbox{orange}{$#1$}}}   % TODO: kleur aanpassen voor mathjax; wordt overschreven infra!

\ifdefined\HCode
\newcommand{\nl}{}
\else
\newcommand{\nl}{\ \par} % newline (achter heading van definition etc.)
\fi

%definities nieuwe commando's - afkortingen veel gebruikte symbolen
\newcommand{\R}{\ensuremath{\mathbb{R}}}
\newcommand{\Rnul}{\ensuremath{\mathbb{R}_0}}
\newcommand{\Reen}{\ensuremath{\mathbb{R}\setminus\{1\}}}
\newcommand{\Rnuleen}{\ensuremath{\mathbb{R}\setminus\{0,1\}}}
\newcommand{\Rplus}{\ensuremath{\mathbb{R}^+}}
\newcommand{\Rmin}{\ensuremath{\mathbb{R}^-}}
\newcommand{\Rnulplus}{\ensuremath{\mathbb{R}_0^+}}
\newcommand{\Rnulmin}{\ensuremath{\mathbb{R}_0^-}}
\newcommand{\Rnuleenplus}{\ensuremath{\mathbb{R}^+\setminus\{0,1\}}}
\newcommand{\N}{\ensuremath{\mathbb{N}}}
\newcommand{\Nnul}{\ensuremath{\mathbb{N}_0}}
\newcommand{\Z}{\ensuremath{\mathbb{Z}}}
\newcommand{\Znul}{\ensuremath{\mathbb{Z}_0}}
\newcommand{\Zplus}{\ensuremath{\mathbb{Z}^+}}
\newcommand{\Zmin}{\ensuremath{\mathbb{Z}^-}}
\newcommand{\Znulplus}{\ensuremath{\mathbb{Z}_0^+}}
\newcommand{\Znulmin}{\ensuremath{\mathbb{Z}_0^-}}
\newcommand{\C}{\ensuremath{\mathbb{C}}}
\newcommand{\Cnul}{\ensuremath{\mathbb{C}_0}}
\newcommand{\Cplus}{\ensuremath{\mathbb{C}^+}}
\newcommand{\Cmin}{\ensuremath{\mathbb{C}^-}}
\newcommand{\Cnulplus}{\ensuremath{\mathbb{C}_0^+}}
\newcommand{\Cnulmin}{\ensuremath{\mathbb{C}_0^-}}
\newcommand{\Q}{\ensuremath{\mathbb{Q}}}
\newcommand{\Qnul}{\ensuremath{\mathbb{Q}_0}}
\newcommand{\Qplus}{\ensuremath{\mathbb{Q}^+}}
\newcommand{\Qmin}{\ensuremath{\mathbb{Q}^-}}
\newcommand{\Qnulplus}{\ensuremath{\mathbb{Q}_0^+}}
\newcommand{\Qnulmin}{\ensuremath{\mathbb{Q}_0^-}}

\newcommand{\perdef}{\overset{\mathrm{def}}{=}}
\newcommand{\pernot}{\overset{\mathrm{notatie}}{=}}


\DeclareMathOperator{\dom}{dom}     % domein
\DeclareMathOperator{\codom}{codom} % codomein
\DeclareMathOperator{\bld}{bld}     % beeld
\DeclareMathOperator{\graf}{graf}   % grafiek
\DeclareMathOperator{\rico}{rico}   % richtingcoëfficient
\DeclareMathOperator{\co}{co}       % coordinaat
\DeclareMathOperator{\gr}{gr}       % graad
%
% Overschrijf addhoc commando's
%
\ifdefined\isEn
\renewcommand{\pernot}{\overset{\mathrm{notation}}{=}}
\RedeclareMathOperator{\bld}{im}     % beeld
\RedeclareMathOperator{\graf}{graph}   % grafiek
\RedeclareMathOperator{\rico}{slope}   % richtingcoëfficient
\RedeclareMathOperator{\co}{co}       % coordinaat
\RedeclareMathOperator{\gr}{deg}       % graad

% Operators
\RedeclareMathOperator{\bgsin}{arcsin}
\RedeclareMathOperator{\bgcos}{arccos}
\RedeclareMathOperator{\bgtan}{arctan}
\RedeclareMathOperator{\bgcot}{arccot}
\RedeclareMathOperator{\bgsinh}{arcsinh}
\RedeclareMathOperator{\bgcosh}{arccosh}
\RedeclareMathOperator{\bgtanh}{arctanh}
\RedeclareMathOperator{\bgcoth}{arccoth}

\fi
\addPrintStyle{..}
%%
\hintstrue

\outcome{Answer some questions about some stuff.}
\outcome{Do something productive.}
\outcome{Learn how Questions work maybe.}
\outcome{Maybe other stuff - who knows!}


\begin{document}
    \xmtitle{Questions}{Simple example of exercise and question}
    
    \displayOutcomes

    
    \begin{exercise} Solve the following questions correctly:
        \begin{question}
                $1+1 = \answer{2}$
        \end{question}    
        \begin{question}
            $1+2 = $ \wordChoice{   \choice{$2$}
                                    \choice[correct]{$3$}
                                    \choice{$4$}
                                }
            \begin{hint}
                By definition, $2 = 1+1$, and addition is associative.
            \end{hint}
            \begin{hint}
                By definition, $1+(1+1) = (1+1) + 1 = 2 + 1 = ?$.
            \end{hint}
            \begin{feedback}[correct] 
                Congratulations, you can already do arithmetic very well! Keep it up. 
            \end{feedback}         
            \begin{oplossing} It holds that
                \[ 1 + 2 = 1 + (1 + 1) = (1 + 1) + 1 = 2 + 1 = 3 \]
                because of the definition $2\perdef 1+1$, the associativity of addition,
                and finally the definition $3\perdef 2+1$. 
            \end{oplossing}
        \end{question}

        \begin{question}$x-4 = 0
            \iff x = \answer{4}$
        \end{question}
        \begin{question}$x-4 = 0
            \iff x = \answer[onlineshowanswerbutton]{4}$
        \end{question}
        \begin{question}$x-4 = 0
            \iff x = \answer[onlinenoinput]{4}$
        \end{question}
        \begin{question}$x^2-4 < 0
            \iff \answer[onlinenoinput]{ x \in \left]-2,2\right[}$
        \end{question}

    \end{exercise}
    \begin{exercise} Tests with answer and feedback

        \begin{question}Write out $3: \answer{drie}$  (using \verb|\answer{drie}|)
        \end{question}

        \begin{question}Write out $3: \answer[format=string]{drie}$ (using \verb|\answer[format=string]{drie}|)
        \end{question}

        \begin{question}Write out $2$: $\answer[id=ansr,format=string]{twee}$ (with tests for feedback)
            \begin{feedback}[correct]{Correct}\end{feedback}
            \begin{feedback}[incorrect]{Incorrect}\end{feedback}    % !!! THIS DOES NOT WORK !!!!
            \begin{feedback}[ansr.toLowerCase() === 'twee']{Correct (because Ximera compares case-insensitive)}\end{feedback}  % but this is equivalent to [correct] ...
            \begin{feedback}[ansr === 'twee']{Correct (according to === in JS)}\end{feedback}  % but this is equivalent to [correct] ...
            % \ifonline{
            % \begin{feedback}[(ansr.toLowerCase() === 'twee') && (ansr !== 'twee')]{Correct (because Ximera compares case-insensitive) }\end{feedback}   % and this works !!!
            % }
            % {
            % \begin{feedback}[]{Correct (because Ximera compares case-insensitive) }\end{feedback}   % NO && in PDF ...?  
            % }
            \begin{feedback}[ansr < 'twee']{Too small}\end{feedback} % this also works, but is probably not very useful 
            \begin{feedback}[ansr > 'twee']{Too big}\end{feedback}
        \end{question}

        \begin{question}
            $\frac{1}{3} =  \answer[tolerance=0.05]{0.33}$  

            Use \verb|\answer[tolerance=0.05]{0.33}| to tolerate rounding errors. You can be off by 0.05 in this case.
        \end{question}

        \begin{question}
            $\frac{1}{3} =  \answer[tolerance=0.05]{\frac{1}{3}}$  

            Use \verb|\answer[tolerance=0.05]{1/3}| to tolerate rounding errors. You can be off by 0.05 in this case.
        \end{question}

        % \begin{question}
        %     $\frac{1}{3} =  \answer[tolerance=0.05]{0.33}$  

        %     Use \verb|\answer[tolerance=0.05]{0.33}| to tolerate rounding errors. You can be off by 0.05 in this case.
        % \end{question}        
        % \begin{question}
        %     $\frac{1}{3} =  \answer[tolerance=0.05]{0,33}$  

        %     Use \verb|\answer[tolerance=0.05]{0.33}| to tolerate rounding errors. You can be off by 0.05 in this case.
        % \end{question}

        \begin{question} Write $x^2: x^{\answer{2}}$
        \end{question}


    \end{exercise}

    \begin{example} Explore which $x\in \R$ satisfy $\left|\dfrac{2x}{3}\right|<1$.

        % \begin{basicSkip}
        Solution set: $\answer[onlinenoinput]{\displaystyle
            V=\left]-\frac{3}{2},\frac{3}{2}\right[}$
        % \end{basicSkip}
        \begin{oplossing}
            None
        \end{oplossing}
    \end{example}

\end{document}